%%%%%%%%%%%%%%%%%%%%%%%%%%%%%%%%%%%%%%%%%
% University/School Laboratory Report
% LaTeX Template
% Version 3.1 (25/3/14)
%
% This template has been downloaded from:
% http://www.LaTeXTemplates.com
%
% Original author:
% Linux and Unix Users Group at Virginia Tech Wiki 
% (https://vtluug.org/wiki/Example_LaTeX_chem_lab_report)
%
% License:
% CC BY-NC-SA 3.0 (http://creativecommons.org/licenses/by-nc-sa/3.0/)
%
%%%%%%%%%%%%%%%%%%%%%%%%%%%%%%%%%%%%%%%%%

%----------------------------------------------------------------------------------------
%	PACKAGES AND DOCUMENT CONFIGURATIONS
%----------------------------------------------------------------------------------------

\documentclass{article}

\usepackage{graphicx} % Required for the inclusion of images
\usepackage{natbib} % Required to change bibliography style to APA
\usepackage{amsmath} % Required for some math elements 
\usepackage{polski}
\usepackage[utf8]{inputenc}
\setlength\parindent{0pt} % Removes all indentation from paragraphs

\renewcommand{\labelenumi}{\alph{enumi}.} % Make numbering in the enumerate environment by letter rather than number (e.g. section 6)

%\usepackage{times} % Uncomment to use the Times New Roman font

%----------------------------------------------------------------------------------------
%	DOCUMENT INFORMATION
%----------------------------------------------------------------------------------------

\title{Stacja zliczająca przejeżdzające rowery \\ przy użyciu STM32 F3 Discovery \\ Systemy Wbudowane 2019} % Title

\date{23 X 2019} % Date for the report

\begin{document}

\maketitle % Insert the title, author and date

\begin{center}
Marcin Pajkowski 211968 (Kierownik)\\ % Partner names
Bartosz Myśliwiec 211827 \\
\end{center}

% If you wish to include an abstract, uncomment the lines below
% \begin{abstract}
% This paper describes 
% \end{abstract}

%----------------------------------------------------------------------------------------
%	Listę wykorzystanych funkcjonalności;
%----------------------------------------------------------------------------------------

\section{Lista wykorzystanych funkcjonalności}

Protokoły wejścia/wyjścia:
\begin{itemize}
    \item General-purpose I/O,
    \item Serial Peripheral Interface,
    \item Nested Vectorized Interrupt Controller,
    \item Universal Asynchronous Receiver-Transmitter.
\end{itemize}

Urządzenia zewnętrzne:
\begin{itemize}
    \item znajdujące się na MCU diody LED,
    \item wyświetlacz,
    \item czujnik zbliżeniowy typu PIR,
    \item moduł Bluetooth.
\end{itemize}

 
%----------------------------------------------------------------------------------------
%	Zakres obowiązków każdego z Autorów (czyli co zrobił?) 
% 	wraz z zaproponowanym przez kierownika projektu udziałem  %	pracy.
%----------------------------------------------------------------------------------------

\newpage
\section{Zakres obowiązków}

\begin{tabular}{|l|l|}
    \hline
    Osoba odpowiedzialna & Moduł funkcjonalny\\
    \hline
    Marcin Pajkowski
    & Przygotowanie środowiska pracy (IDE, repozytorium Git)\\
    & Komunikacja UART (serial.h, serial.c)\\
    & Sygnalizacja za pomocą diod (led.h, led.c)\\
    & Moduł logujący (trace.h)\\
    & Czujnik ruchu (motion.h, motion.c)\\
    \hline
    Bartosz Myśliwiec
    & RTC\\
    & Wyświetlacz (display.h, display.c)\\
    & Czujnik temperatury (display.h, display.c)\\
    & Przycisk (button.h, button.c)\\
    \hline
\end{tabular}

%----------------------------------------------------------------------------------------
%	Opis działania programu 
%----------------------------------------------------------------------------------------

\section{Opis działania programu}
\subsection{Instrukcja użytkownika}

Urządzenie umożliwi wymianę komunikatów za pomocą interfejsu UART. Odbiorca urządzenia
musi upewnić się, że jego komputer posiada możliwość komunikacji. W systemach
z rodziny GNU/Linux może okazać się konieczne posiadanie uprawnienia do tworzenia
procesów jako użytkownik uprzywilejowany lub przynależność do grupy dialout. Zalecaną opcją
jest sparowanie produktu za pomocą Bluetooth. Jeśli użytkownik
nie posiada wyżej wymienionego modułu w swojej maszynie może spróbować użyć
dowolnego urządzenia zgodnego z UART zamiast modułu HC-05 dołączonego wraz z projektem.

Hasło do Bluetooth jest zamieszczone na urządzeniu.

Aby przetestować połączenie z komputer należy wysłać znak '+'. Odpowiedź "OK" oznacza, że wszystko zostało
skonfigurowane pomyślnie.

\subsection{Opis algorytmu}
Przy tworzeniu oprogramowania zostało zastosowane 

%----------------------------------------------------------------------------------------
%	Opis działania wykonanego sprzętu wraz z notami 	  	katalogowymi (w dodatkowych plikach) z wyjaśnieniem dlaczego zrobione jest tak, a nie inaczej.
%Jeżeli nie było wykonanego sprzętu, to trzeba napisać, że %nie było;
%----------------------------------------------------------------------------------------

\section{Opis działania wykonanego sprzętu\\ wraz z notami katalogowymi}

\subsection{Wyświetlacz }

%----------------------------------------------------------------------------------------
%Opis funkcjonalności poszczególnych elementów, np. dla 	wejść wyjść cyfrowych w układach LPC należy opisać PINSEL (w używanym zakresie), IODIR, IOSET, IOCLR, IOPIN i tak dla wszystkich używanych funkcjonalności [UART, SPI, I2C, PWM, ...] i interfejsów [Wyświetlacz, karta MMC/SD, Extender, karta sieciowa, akcelerometr, ...], czy przerwań (tych które są używane, dla LPC2xxx - FIQ,VIRQ,IRQ).
%----------------------------------------------------------------------------------------

\section{Opis funkcjonalności poszczególnych \\ elementów}
...


%----------------------------------------------------------------------------------------
%	BIBLIOGRAPHY
%----------------------------------------------------------------------------------------

\bibliographystyle{apalike}

\bibliography{sample}

%----------------------------------------------------------------------------------------


\end{document}