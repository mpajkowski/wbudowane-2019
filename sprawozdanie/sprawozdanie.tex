%%%%%%%%%%%%%%%%%%%%%%%%%%%%%%%%%%%%%%%%%
% University/School Laboratory Report
% LaTeX Template
% Version 3.1 (25/3/14)
%
% This template has been downloaded from:
% http://www.LaTeXTemplates.com
%
% Original author:
% Linux and Unix Users Group at Virginia Tech Wiki 
% (https://vtluug.org/wiki/Example_LaTeX_chem_lab_report)
%
% License:
% CC BY-NC-SA 3.0 (http://creativecommons.org/licenses/by-nc-sa/3.0/)
%
%%%%%%%%%%%%%%%%%%%%%%%%%%%%%%%%%%%%%%%%%

%----------------------------------------------------------------------------------------
%	PACKAGES AND DOCUMENT CONFIGURATIONS
%----------------------------------------------------------------------------------------

\documentclass{article}

\usepackage{graphicx} % Required for the inclusion of images
\usepackage{natbib} % Required to change bibliography style to APA
\usepackage{amsmath} % Required for some math elements 
\usepackage{polski}
\usepackage{hyperref}
\usepackage{listings}
\usepackage{sidecap}
\usepackage{outlines}
\graphicspath{ {./images/}}
\usepackage[utf8]{inputenc}
\setlength\parindent{0pt} % Removes all indentation from paragraphs

\renewcommand{\labelenumi}{\alph{enumi}.} % Make numbering in the enumerate environment by letter rather than number (e.g. section 6)

%\usepackage{times} % Uncomment to use the Times New Roman font

%----------------------------------------------------------------------------------------
%	DOCUMENT INFORMATION
%----------------------------------------------------------------------------------------

\title{Stacja zliczająca przejeżdzające rowery \\ przy użyciu STM32 F3 Discovery \\ Systemy Wbudowane 2019} % Title

\date{23 X 2019} % Date for the report

\begin{document}

\maketitle % Insert the title, author and date

\begin{center}
Marcin Pajkowski 211968 (Kierownik)\\ % Partner names
Bartosz Myśliwiec 211827 \\
\end{center}

% If you wish to include an abstract, uncomment the lines below
% \begin{abstract}
% This paper describes 
% \end{abstract}

%----------------------------------------------------------------------------------------
%	Listę wykorzystanych funkcjonalności;
%----------------------------------------------------------------------------------------

\section{Lista wykorzystanych funkcjonalności}

Protokoły wejścia/wyjścia:
\begin{itemize}
    \item General-purpose I/O,
    \item Serial Peripheral Interface,
    \item Nested Vectorized Interrupt Controller,
    \item Universal Asynchronous Receiver-Transmitter.
\end{itemize}

Urządzenia zewnętrzne:
\begin{itemize}
    \item znajdujące się na MCU diody LED,
    \item wyświetlacz,
    \item czujnik zbliżeniowy typu PIR,
    \item moduł Bluetooth.
\end{itemize}

 
%----------------------------------------------------------------------------------------
%	Zakres obowiązków każdego z Autorów (czyli co zrobił?) 
% 	wraz z zaproponowanym przez kierownika projektu udziałem  %	pracy.
%----------------------------------------------------------------------------------------

\newpage
\section{Zakres obowiązków}

\begin{tabular}{|l|l|}
    \hline
    Osoba odpowiedzialna & Moduł funkcjonalny\\
    \hline
    Marcin Pajkowski (50\%)
    & Komunikacja UART (serial.h, serial.c)\\
    & Wyświetlacz (display.h, display.c)\\
    & Moduł logujący (trace.h)\\
    & Czujnik ruchu (motion.h, motion.c)\\
    & Komendy (shell.h, shell.c)\\
    \hline
    Bartosz Myśliwiec (50\%)
    & RTC (rtc.h, rtc.c)\\
    & Sygnalizacja za pomocą diod (led.h, led.c)\\
    & Czujnik temperatury (adc.h, adc.c)\\
    & Przycisk (button.h, button.c)\\
    \hline
\end{tabular}

%----------------------------------------------------------------------------------------
%	Opis działania programu 
%----------------------------------------------------------------------------------------

\section{Opis działania programu}
\subsection{Instrukcja użytkownika}

Urządzenie umożliwi wymianę komunikatów za pomocą interfejsu UART. Odbiorca urządzenia
musi upewnić się, że jego komputer posiada możliwość komunikacji. W systemach
z rodziny GNU/Linux może okazać się konieczne posiadanie uprawnienia do tworzenia
procesów jako użytkownik uprzywilejowany lub przynależność do grupy dialout.

Od momentu włączenia/resetu wysyłane jest do portu szeregowego informacja o zarejestrowanym
ruchu. Ramka danych ma następującą postać:

\begin{lstlisting}
    +DATABEGIN+[data];[czas]+DATAEND+\r\n
\end{lstlisting}

przy czym pole \emph{data} ma format \emph{DD.MM.YYYY}, a pole czas - \emph{HH:MM:SS} (format 24-godzinny).

Zastosowanie ramki umożliwia odfiltrowanie informacji o zdarzeniu ruchu od logów, które
są wysyłane za pomocą tego samego UART.

\subsection{Korzystanie z komend powłoki}
Za pomocą interfejsu UART możemy wysyłać komendy do rejestratora ruchu. Ogólna postać komendy
prezentuje się następująco:

\begin{lstlisting}
    komenda [argumenty,...]
\end{lstlisting}

Wprowadzone dane należy zatwierdzić używając znaku '='.

\subsubsection{Ustawianie daty}
W celu ustawienia daty należy użyć komendy \emph{setDate} w następujący sposób:

\begin{lstlisting}
    setDate DD,MM,YYYY
\end{lstlisting}

\subsubsection{Ustawianie czasu}
W celu ustawienia czasu należy użyć komendy \emph{setTime} w następujący sposób:

\begin{lstlisting}
    setTime HH,MM
\end{lstlisting}

\subsubsection{Pobieranie informacji o aktualnym czasie}
Aby otrzymać informacje o aktualnym czasie należy użyć komendy \emph{getTime}:

\begin{lstlisting}
    getTime
\end{lstlisting}

\subsubsection{Pobieranie informacji o aktualnej dacie}
Aby otrzymać informacje o aktualnej dacie należy użyć komendy \emph{getDate}:

\begin{lstlisting}
    getDate
\end{lstlisting}

% TODO - przykłady odpowiedzi rejestratora

\subsection{Opis algorytmu}

\subsubsection{Ogólny opis algorytmu}
%% kolejność jest z main.c
Po włączeniu urządzenia wywoływane są funkcje inicjalizujące moduły funkcjonalne
urządzenia. Podczas tego procesu konfigurowane i włączane zostają porty GPIO,
peryferia odpowiadające za I/O, kontrolery przerwań oraz zegary.

\paragraph{Diody}
W pierwszej kolejności następuje inicjalizacja pinów portu GPIOE - do wyjść
są podłączone diody znajdujące się na module głównym. Proces ten prezentuje się
następująco:

\begin{enumerate}
    \item włączenie linii zegarowej AHB1,
    \item wybranie pinów pinów 8-15 jako piny aktywne,
    \item ustawienie wybranych pinów w tryb wyjścia.
\end{enumerate}

\paragraph{Wyświetlacz}
Następnym krokiem jest inicjalizacja wyświetlacza. Podobnie jak w przypadku diod, najpierw
należy zainicjalizować GPIO. Należy jednak poinformować układ, że 
będziemy korzystać z protokołu SPI. W przypadku tego procesora 
taką informację konfiguruje się w czasie uruchomienia poprzez ustawienie
trybu funkcji alternatywnych oraz wskazania go.
Dla pinów \emph{DIN} oraz \emph{CLK} wybieramy tryb \emph{AF5}, wybór ten
oznacza, że naszą funkcją alternatywną z kótrej będziemy korzystać jest \emph{SPI1}.

Pozostałe piny powiązane z wyświetlaczem (\emph{DC}, \emph{CE}, \emph{RST})
ustawiamy w tryb wyjścia oraz ustawiamy na nich stan wysoki.

\paragraph{Czujnik ruchu}
Ustawiamy pin podłączony do czujnika ruchu jako pin wejściowy. Czujnik będzie ustawiał
stan wysoki gdy ruch zostanie wykryty i stan niski w sytuacji przeciwnej.
Do obserwacji stanu tego pinu zostało użyte przerwanie \emph{EXTI3}.
Inicjalizacja EXTI przebiega następująco:

\begin{itemize}
    \item Wybranie linii 3,
    \item ustawienie trybu EXTI jako przerwanie,
    \item ustawienie wyzwalacza jako \emph{rising-falling}
        (oznacza to, że przerwanie zostanie wyzwolone przy zmianie stanu pinu),
    \item włączenie linii komend
\end{itemize}
Następnie włączamy przerwanie za pomocą \emph{NVIC}.

\paragraph{UART}
Inicjalizacja portu szeregowego UART zaczyna się od skonfigurowania pinów
odpowiadających za nadawanie/odczyt. Wybieramy funkcję alternatywną \emph{AF7}.
Inicjalizujemy napierw piny \emph{GPIOA}, następnie urządzenie \emph{USART1}.
Należy jeszcze poczekać na ustawienie flag \emph{TEACK} i \emph{REACK}. Ostatnim krokiem
jest właczenie przerwania IRQ dla \emph{USART1}. "Właścicielem" tego przerwania jest usługa komend.


\paragraph{Przycisk}
Kolejnym inicjalizowanym urządzeniem peryferyjnym jest przycisk znajdujący
się na module. Jest on połączony z pinem 0 portu GPIOA. Dla tego pinu ustawiamy tryb
wejścia. Po naciśnięciu guzika są włączane diody - jest to zrealizowane za pomocą przerwania \emph{EXTI}

\paragraph{RTC}
Inicjalizacje RTC rozpoczynamy od włączenia zewnętrznego zegara wysokiej częstotliwości HSE.
RTC może być taktowane z zegarów HSE, LSE lub LSI. Zdecydowaliśmy się na zegar HSE ponieważ
płytka STM32F3DISCOVERY nie jest fabrycznie wyposażona w oscylator kwarcowy zapewniający
taktowanie dla LSE, a taktowanie z LSI okazało się być nieodpowiednie dla zegara RTC, zegar
taktowany tym zegarem przyśpieszał lub spowalniał co powodowało desynchronizacje i utratę informacji
o czasie jaki upłynął. Zegar HSE posiada zbyt dużą częstotliwość taktowania dlatego konieczne jest
skorzystanie z dzielnika.
Żeby włączyć zegar HSE korzystamy z peryferium RCC które kontroluje inne peryferia w tym zegar
HSE.
Następnie czekamy aż RCC zgłosi gotowość do działania HSE poprzez ustawienie bitu HSERDY.

Kolejnym krokiem jest włączenie zegara PWR i zdjęcie blokady dostępu do backupu w którym jest
przechowywana konfiguracje źródła zegara RTC.
Ustalamy źródło taktowania zegara RTC jako HSE z dzielnikiem o wartości 32.
Włączamy RTC ustawiając bit RTCEN w rejestrze BDCR peryferium RCC.
Rejestry RTC są chronione przed zmianą, żeby móc je edytować konieczne jest wyłączenie blokady, w
tym celu wpisujemy do rejestru WPR sekwencje 0xCA, 0x53. Od teraz rejestry RTC mogą być
edytowane.

Włączamy tryb inicjalizacji.
Ustawiamy wartości preskalowania synchronicznego i asynchronicznego. Preskalowanie ma na celu 
zapewnienie taktowania $ ck_spre $ równe \emph{1Hz}.

Ponieważ w naszym przypadku wybraliśmy zegar
HSE dzielony przez 4. Ustaliliśmy preskaler asynchroniczny na 124, a preskaler synchroniczny na 1999.
$$ 250000 / (125*2000) = 1Hz $$
Następnie opuszczamy tryb inicjalizacji, blokujemy możliwość edycji rejestrów RTC oraz blokujemy
dostęp do backup’u.

Następnym etapem jest ustawienie czasu. Włączamy możliwość edycji backup’u, wyłączamy blokadę
edycji rejestrów RTC, wchodzimy w tryb inicjalizacji. Umieszczamy czas w rejestrze TR oraz informacje
że chcemy reprezentować czas w formacie 24 godzinnym.
Wychodzimy z trybu inicjalizacji, Wyłączamy możliwość edycji backup’u, włączamy blokadę edycji
rejestrów RTC.
Kolejnym etapem jest ustawienie daty. Włączamy możliwość edycji backup’u, wyłączamy blokadę edycji
rejestrów RTC, wchodzimy w tryb inicjalizacji. Umieszczamy datę i dzień tygodnia w rejestrze DR.
Wychodzimy z trybu inicjalizacji, Wyłączamy możliwość edycji backup’u, włączamy blokadę edycji
rejestrów RTC.
Dokonujemy inicjalizacji alarmu. Włączamy dostęp do backup’u, dezaktywujemy alarm A by móc go
ponownie skonfigurować, czekamy aż flaga ALRAW przestanie być ustawiona.
Inicjalizujemy alarm z maską ustawioną na wszystkie pola rejestru, oznacza to że alarm będzie
uruchamiany co sekundę. Włączamy ponownie dostęp do backupu ponieważ funkcja biblioteczna
automatycznie go wyłącza, aktywujemy alarm, blokujemy dostęp do backup’u.
Ostatnim krokiem jest włączenie generowania przerwania za każdym razem gdy zostanie ustawiona flaga
oznaczająca wystąpienie alarmu A.
Włączamy dostęp do backup’u, włączamy alarm A, aktywujemy przerwanie na linii 17 kontrolera

przerwań EXTI, ustalamy generowanie przerwania na rosnącym zboczu. Ustawiamy priorytet 15.
aktywujemy obsługę alarmu, blokujemy dostęp do backup’u.
Obsługa przerwania polega na odblokowania dostępu do backup’u wyczszczenia flagi alarmu,
wyczyszczeniu flagi przerwania EXTI, pobranie godziny, wykonanie zaplanowanych rutyn,
zablokowanie dostępu do backup’u.

\paragraph{Przetwornik analogowo-cyfrowy}
Przetworniki analogowo-cyfrowe to układy peryferyjne które są
zdolne do przetwarzania wejściowej wartości napięcia w postaci analogowej
na reprezentacje w postaci cyfrowej, dzięki czemu można korzystać z dyskretyzowanych pomiarów
wartości które z natury są analogowe takie jak: temperatura, napięcie, wilgotność itp.
Mikrokontroler w który wyposażona jest płytka STM32F3DISCOVERY posiada cztery
przetworniki analogowo-cyfrowe. Są to przetworniki typu SAR(successive-approximation-register)
Pierwszym krokiem jest więc włączenie zegara ADC. Wybraliśmy taktowanie zegara szyny
AHB z systemowym dzielnikiem przez 2.
ADC w tym mikrokontrolerze może wykonywać pomiary w trybie pojedynczego pomiaru(one
shot) jednego kanału bądź ich sekwencji. Tryb ten może być używany na przykład do sprawdzenia
warunków koniecznych do uruchomienia pozostałych elementów systemu, na przykład podjęciu decyzji
czy bateria posiada odpowiedni napięcie. Po zakończeniu pomiaru przetwornik wymaga ponownego
skonfigurowania by wykonać następny pomiar.
Ponieważ chcemy wyświetlać bieżącą temperaturę korzystamy z innego trybu, a mianowicie
trybu Continuous, który wykonuje następny pomiar od razu po zakończeniu pierwszego pomiaru bądź
sekwencji pomiarów. Nie jest konieczne ponowne konfigurowanie i aktywowanie przetwornika. W celu
użycia trybu Continuous ustawiamy w rejestrze CFGR przetwornika ADC1
bit CONT na 1.
ADC może wykonywać pomiar jednego kanału, bądź sekwencje pomiarów różnych kanałów.
Do wykonania kilku pomiarów(scan mode) należy ustalić ich ilość w rejestrze SQR1.
Ponieważ wykonujemy pomiar tylko jednego kanału, ustawiamy w rejestrze SQR1 bity L na 0.
Kolejnym krokiem jest ustalenie kolejności kanałów które będą mierzone, ponieważ używamy
tylko jednego kanału dokonujemy tylko jego konfiguracji i ustalamy że jego pomiar ma być wykonywany
jako pierwszy. Ponieważ odczyt następuje z portu GPIO który jest podłączony do ósmego kanału ADC1
wpisujemy wartość go reprezentującą w bity SQ1 rejestru SQR1.
Następnie ustalamy co ile cykli ADC ma być wykonywany pomiar na danym kanale,
zdecydowaliśmy się na użycie najrzadszego wykonywania pomiarów czyli co 601.5 cykla ADC. Częstsze
wykonywanie pomiarów nie jest potrzebne ponieważ wyświetlamy wynik co 1 sekundę.
By skonfigurować częstotliwość pomiarów na 601.5 cykla ADC wpisujemy 7 w bity SMP1 rejestru
SMPR1.
Możliwe wartości rozdzielczości to 6, 8, 10 oraz 12 bitów na pomiar. W naszej aplikacji
przetwornika ADC w celu zapewnienia najbardziej dokładnych pomiarów zdecydowaliśmy się na
wykorzystanie maksymalnej rozdzielczości – 12 bitowej.
Dokonujemy ustawienia rozdzielczości wykonywanych pomiarów w tym celu ustawiamy w

rejestrze CFGR przetwornika ADC1 bity RES na 0 co informuje przetwornik że chcemy otrzymywać
wynik konwersji w postaci 12 bitowej.
Dla 12 bitowej rozdzielczości występuje $$ 2^{12} = 4096 $$ stopni dyskretyzacji, oznacza to że w
naszym przypadku wartość pomiaru mieści się w przedziale od 0 dla wartości bliskich braku mierzonego
napięcia do 4095 dla wartości mierzonego napięcia bliskich wartości napięcia referencyjnego bądź od
niego większych. Ponieważ napięcie referencyjne w płytce STM32F3DISCOVERY wynosi 3V oznacza
to że zakres pomiarowy mieści się od 0V do 3V.
Następnym krokiem jest określenie czy wynik konwersji ma być wyrównany do lewej czy
prawej. Zdecydowaliśmy się na wyrównanie do prawej, ponieważ taki rodzaj wyrównania nie wymaga
późniejszej konwersji programowej w celu uzyskania wyniku. Wyrównanie do prawej oznacza ze
najbardziej znaczące bity RDATA w rejestrze DR są dopełniane zerami.
By uzyskać wyrównanie do prawej ustawiamy w rejestrze CFGR przetwornika ADC1 bit ALIGN
na wartość 0.
Ponieważ wyniki chcemy pobierać po każdej zakończonej konwersji, włączamy generowanie
przerwania po zakończeniu sekwencji konwersji. W tym celu ustawiamy bit EOSIE na 1 w rejestrze IER.
Następnie aktywujemy przetwornik ADC1 ustawiając w rejestrze kontrolnym CR bit ADEN na
jeden i czekamy aż przetwornik zgłosi gotowość do pracy ustawiając flagę ADRDY w rejestrze ISR.
Ostatni krok to zarejestrowanie przerwanie w NVIC oraz rozpoczęcie konwersji ustawiając w
rejestrze kontrolnym CR wartość ADSTART na 1.
Od teraz po każdej skończonej konwersji występuje przerwanie, przerwanie to jest obsłużone w ten
sposób że umieszczamy wynik konwersji w zmiennej pełniącej funkcję akumulatora a na koniec
ustawiamy flagę EOS na 0, po 500 konwersjach wyliczamy średnią arytmetyczną i zmieniamy zmienna
globalną reprezentującą średnią wartość konwersji, zabieg ten ma na celu wyeliminowanie szumu jaki
może wpływać na wynik pojedynczej konwersji.
Kluczowym czynnikiem wpływającym na dokładność pomiaru jest stała wartość napięcia
referencyjnego. Jest to spowodowane tym że zasada działania przetwornika opiera się na porównywaniu
wartości mierzonego napięcia z wartością napięcia referencyjnego, wynik ten wyrażany jest w dyskretnej
skali której przedział zaczyna się od zera a maksimum jest zależne od rozdzielczości wykonywanych
pomiarów która jest wyrażona w bitach na pomiar.
W celu zmierzenia temperatury otoczenia wykorzystaliśmy termometr KTY81-210 jest to
krzemowy czujnik temperatury oparty na zasadzie działania termistora, to znaczy że rezystancja elementu
zmienia się wraz z jego temperaturą. Zaletami tego czujnika są: niska cena, brak polaryzacji, podłączenie
wymaga tylko dwóch pinów. Wadami zaś jest niska dokładność pomiarów.

Ponieważ czujnik KTY81-210 wyraża temperaturę jako rezystancję, a przetworniki analogowo-cyfrowe
mikroprocesora STM32F3DISCOVERY są zdolne jedynie do mierzenia napięcia, konieczne jest
zastosowanie dzielnika napięcia.

W tym celu zrealizowano układ dzielnika napięcia przedstawiony na poniższym schemacie.

\paragraph{Zadania}
Poniższe zadania zostały zrealizowane za pomocą przerwań.

\subparagraph{Obserwowanie stanu pinu czujnika}
Od momentu inicjalizacji przerwania \emph{EXTI3} urządzenie oczekuje na
komunikat z czujnika PIR o wykrytym ruchu, czyli zmianę stanu pinu wejściowego
czujnika.

Możliwe akcje:
\begin{itemize}
    \item Zmiana stanu na wysoki\newline
    Urządzenie wysyła ramkę z datą i czasem za pomocą \emph{UART1}. Wszystkie diody LED
    zaczynają świecić (ustawienie stanu wysokiego na wybranych pinach portu \emph{GPIOE})

    \item Zmiana stanu na niski\newline
    Urządzenie wyłacza diody
\end{itemize}

Dodatkowe efekty uboczne:
Jeśli oprogramowanie zostało skompilowane w trybie \emph{DEBUG} otrzymamy szczegółowe
logi z informacjami o zmianach stanu.

\subparagraph{Wykonywanie komend}
Zaraz po inicjalizacji przerwania \emph{USART1} urządzenie uruchamia wyzwalacz związany
z obsługą komend;

Możliwe akcje:
\begin{itemize}
   \item Wprowadzono znak niebędący '=' lub '\b'\newline
   Urządzenie zapisuje w buforze komendy otrzymany znak, inkrementuje też
   kursor komendy
   \item Wprowadzono znak '\b'\newline
   Urządzenie usuwa ostatni znak z bufora i dekrementuje kursor.
   \item Wprowadzono znak '='\newline
   Urządzenie przekazuje bufor komendy do dalszego procesowania.
\end{itemize}

Dodatkowe efekty uboczne:
Jeśli składnia komendy zostanie zaklasyfikowana przez algorytm jako \emph{ill-formed},
lub z jakiegokolwiek innego powodu wykonanie komendy nie powiedzie się - użytkownik powinien
otrzymać informację w logach (poziom \emph{WARN}). Zostanie zwrócony także odpowiedni komunikat.

Jeśli komenda powiedzie się - odpowiednia informacja zwrotna powinna zostać przekazana
użytkownikowi.

%----------------------------------------------------------------------------------------
%	Opis działania wykonanego sprzętu wraz z notami 	  	katalogowymi (w dodatkowych plikach) z wyjaśnieniem dlaczego zrobione jest tak, a nie inaczej.
%Jeżeli nie było wykonanego sprzętu, to trzeba napisać, że %nie było;
%----------------------------------------------------------------------------------------

\section{Opis działania wykonanego sprzętu wraz z notami katalogowymi}

\subsection{Moduł główny}
% TODO: GPIO, UART, SPI, RTC, EXTI, NVIC
Projekt został zrealizowany z użyciem płytki ewaluacyjnej STM32F3 Discovery.
Układ ten został wyposażony w procesor rodziny Arm® Cortex®-M4 - STM32F303VCT6.

\subsection{Interfejsy modułu głównego wykorzystane w projekcie}
\subsubsection{GPIO}
GPIO (ang. \emph{General Purpose Input/Output}) jest interfejsem ogólnego przeznaczenia, są one grupowane w porty.
Umożliwiają pracę pinów w trybach:

\begin{enumerate}
   \item wejścia,
   \item wyjścia,
   \item funkcji alternatywnej.
\end{enumerate}

\subsubsection{SPI}
SPI (ang. \emph{Serial Peripheral Interface}) jest interfejsem umożliwiającym transmisję
danych z wykorzystaniem portu szeregowego. Transmisja danych odbywa się synchronicznie.
Pomiędzy urządzeniami przesyłany jest sygnał taktujący.

SPI korzysta z 4 linii:
\begin{enumerate}
   \item SS \emph{Slave Select} - wybór urządzenia aktywnego (aktywowana stanem niskim),
   \item MISO \emph{Master Input Slave Output} - dane z węzła pomocniczego do węzła głównego,
   \item MOSI \emph{Master Output Slave Input} - dane z węzła głównego do węzła pomocniczego,
   \item SCLK \emph{Serial Clock} - linia zegara.
\end{enumerate}

\subsubsection{UART}
UART (ang. \emph{Universal Asynchronous Receiver-Transmitter}) podobnie jak SPI służy
do szeregowej transmisji danych.

\paragraph{Ramka UART}
Transmisja danych rozpoczyna się bitem startu (wartość logiczna 0), następnie przesyłane są
bity danych - ich ilość jest konfigurowalna i wynosi odpowiednio 7, 8 lub 9. Opcjonalnie
może pojawić się bit parzystości (ang. \emph{Parity bit}). Koniec ramki jest 
oznaczany bitem o wartości logicznej 1.

\subsubsection{Datasheet}
\url{https://www.st.com/resource/en/data_brief/stm32f3discovery.pdf}
\newline
\url{https://www.st.com/resource/en/datasheet/stm32f303vc.pdf}

\subsubsection{UART i SPI - realizacja asynchronicznego I/O}
Peryferia UART i SPI użytego układu STM32 realizują koncepcję
asynchronicznego I/O. Operacje zapisu i odczytu są nieblokujące, co oznacza, że
wykonanie takiej operacji nie wstrzyma bieżącej jednostki wykonującej kod do momentu
uzyskania informacji zwrotnej - w kontekście wejścia/wyjścia. Gdy operacja związana
z danymi jest kompletna zostają ustawione odpowiednie flagi w rejestrze flagowym.
W wypadku UART takim rejestrem jest \emph{ISR}, a dla SPI - \emph{SR}.

W tym projekcie nie zostały jednak wykorzystane możliwości tego podejścia - po wszystkich
operacjach blokujących następuje czekanie na zmianę statusu danej flagi.

\subsubsection{NVIC}
NVIC (ang. \emph{Nested Vectorized Input Controller}) jest sprzętowym kontrolerem
przerwań układu użytego w projekcie. Za jego pomocą można aktywować/deaktywować
przerwania. Do jego zadań należy też właściwa priorytetyzacja.

\paragraph{Przerwanie}
Przerwanie jest sygnałem zgłaszającym żądanie zmiany przepływu sterowania.
W ogólności można podzielić je na:

\begin{itemize}
    \item przerwania sprzętowe - pochodzące od układu sprzętowego,
    \item przerwania programowe - pochodzące od oprogramowania - przeważnie od
            systemu operacyjnego.
\end{itemize}

W tym projekcie będziemy mieć do czynienia wyłącznie z przerwaniami programowymi.
Te dzielą się na:

\begin{itemize}
    \item przerwania zewnętrzne - pochodzące od zewnętrznych układów,
    \item przerwania wewnętrzne - zgłaszane przez procesor (np dzielenie przez zero).
\end{itemize}

W realizowanym projekcie nie powinny wystąpić przerwania wewnętrzne.
Użyte przerwania zewnętrzne:

\begin{enumerate}
    \item \emph{EXTI} - przerwanie przycisku,
    \item \emph{USART} - przerwanie wyzwalane gdy przyjdą dane z UART,
    \item \emph{RTC\_Alarm} - wyzwalane co sekundę,
    \item \emph{ADC} - wyzwalane po każdej konwersji,
\end{enumerate}

\paragraph{Funkcje zwrotne przerwań}
Akcję funkcji zwrotnej (ang. \emph{callbacks}) należy zaimplementować funkcji
o odpowiedniej sygnaturze i nazwie symbolu. Ich prototypy wyglądają następująco:

\begin{center}
\begin{lstlisting}[language=C, basicstyle=\footnotesize]
void {nazwa_przerwania}_IRQHandler();
\end{lstlisting}
\end{center}

na przykład dla przerwania \emph{USART1} taka funkcja będzie miała sygnaturę

\begin{center}
\begin{lstlisting}[language=C, basicstyle=\footnotesize]
void USART1_IRQHandler();
\end{lstlisting}
\end{center}

Domyślna implementacja takiej funkcji jest pusta, tj. zawiera puste ciało.

\paragraph{Własne implementacje a One Definition Rule}
Biblioteka STM32 definiuje funkcje zwrotne przerwań jako symbole słabe
(ang. \emph{weak aliases}). Definicje takich symboli zostaną nadpisane przez
implementacje dostarczone przez użytkowników biblioteki.

\subsection{Urządzenia zewnętrzne}
\subsubsection{Wyświetlacz PCD8544}

W projekcie został użyty monochromatyczny wyświetlacz ciekłokrystaliczny. Ten sam model
został użyty w telefonie Nokia 5110. Na wyświetlaczu prezentowany jest aktualny czas wraz z datą.
Komunikacja z wyświetlaczem jest zrealizowana za pomocą układu SPI.
Urządzenie w zakupionej przez nas wersji posiada 8 wyprowadzeń:

\begin{enumerate}
    \item RST - linia resetująca wyświetlacza,
    \item CE - linia SS,
    \item DC - flaga trybu komend/danych,
    \item DIN - linia danych,
    \item CLK - linia SCK,
    \item VCC - zasilanie,
    \item BL - zasilanie podświetlenia,
    \item GND - masa.
\end{enumerate}

\paragraph{Adresacja}
Rozdzielczość wyświetlacza to 84x48 pikseli. Ekran podzielony jest na 6 banków.
Taki podział ułatwia pracę z tekstem - po wysłaniu oktetu licznik adresów
X jest inkrementowany. W momencie osiągnięcia granicy wskaźnik X jest ustawiany na
0 a wskaźnik Y inkrementowany o 1. Dzięki takiemu sposobowi adresacji można użyć
jednowymiarowego bufora. W każdym bajcie najstarszy bit oznacza piksel znajdujący się najniżej,
zaś najmłodszy - piksel u góry.


\begin{figure}[ht]
    \includegraphics[scale=0.50]{address_display}
    \caption{Adresacja banków pamięci, V=0}
\end{figure}

\paragraph{Inicjalizacja wyświetlacza}
Zaraz po włączeniu zasilania stany rejestrów i pamięci urządzenia są niezdefiniowane.
Należy wygenerować stan niski na linii RST. Następnie na tej samej linii ustawiany
jest stan wysoki.

\begin{center}
\begin{lstlisting}[language=C, basicstyle=\footnotesize]
void displayReset()
{
    LL_GPIO_ResetOutputPin(DISPLAY_PORT, DISPLAY_RST);
    LL_GPIO_SetOutputPin(DISPLAY_PORT, DISPLAY_RST);
} 
\end{lstlisting}
\end{center}

Następnie należy wysłać serię komend:

\begin{enumerate}
    \item Rozszerzony tryb komend
    \item Ustawienie kontrastu
    \item Współczynnik temperatury
    \item Współczynnik biasu
    \item Powrót do regularnego trybu komend
\end{enumerate}


\paragraph{Wysyłanie komend/danych}
Aby wysłać do wyświetlacza komendę należy ustawić na linii DC stan wysoki. Wysyłanie danych
jest oznaczone stanem niskim. Obydwie operacje odbywają się za pomocą interfejsu SPI.
\paragraph{Datasheet}
\url{https://www.sparkfun.com/datasheets/LCD/Monochrome/Nokia5110.pdf}

\subsubsection{Czujnik ruchu}

\paragraph{Opis}
Jako czujnik ruchu został wybrany czujnik typu PIR. Jest to urządzenie pasywne.
Czujniki tego typu wykrywają szybkie zmiany temperatury w swoim otoczeniu. Czujniki
PIR charakteryzują się wysoką ilością sygnałów \emph{false-positive}, dlatego też
urządzenie zastosowane w projekcie ma możliwość regulacji czułości, które pomaga
w znacznym stopniu zniwelować ten problem. Komunikacja z czujnikiem jest prosta - wystawia
on stan wysoki (+3V) na wyjściu w sytuacji, kiedy ruch zostanie wykryty.

\paragraph{Datasheet}
\url{https://www.mpja.com/download/31227sc.pdf}

\subsubsection{Bluetooth}
Wybrane przez nas urządzenie umożliwia komunikację poprzez UART.

\paragraph{Datasheet}
\url{http://www.electronicaestudio.com/docs/istd016A.pdf}

\section{Opis funkcjonalności poszczególnych elementów}

Przycisk
\begin{itemize}
    \item AHB1
    \item GPIOA
    \item NVIC
    \begin{enumerate}
        \item PIN0 - Stan przycisku
    \end{enumerate}
\end{itemize}

Wyświetlacz
\begin{itemize}
    \item AHB1
    \item APB2
    \item SPI1
    \item GPIOA
        \begin{enumerate}
            \item PIN1 - DC
            \item PIN2 - Podświetlenie
            \item PIN4 - CE
            \item PIN5 - CLK
            \item PIN6 - Reset
            \item PIN7 - DIN
        \end{enumerate}
\end{itemize}

Diody LED
\begin{itemize}
    \item AHB1
    \item GPIOE - PIN8-PIN15 - stany diod LED
\end{itemize}

Czujnik ruchu
\begin{itemize}
    \item AHB1
    \item NVIC
    \item GPIOA
        \begin{enumerate}
            \item PIN3 - Odczyt stanu
        \end{enumerate}
\end{itemize}

Bluetooth
\begin{itemize}
    \item AHB1
    \item APB2
    \item NVIC
    \item USART1
    \item GPIOA
        \begin{enumerate}
            \item PIN9 - USART1 Transmitter
            \item PIN10 - USART1 Receiver
        \end{enumerate}
\end{itemize}

\section{Analiza FMEA}
\begin{table}[]
\begin{tabular}{|l|l|l|l|l|}
\hline
Ryzyko                                                                           & Prawdopodobieństwo & Znaczenie & (Samo)wykrywalność                                                                                                                        & Iloczyn \\
\hline
Uszkodzenie mikrokontrolera                                                      & 0.1                & 10        & \begin{tabular}[c]{@{}l@{}}Brak komunikacji przez\\ port szeregowy. Analiza\\ utraty połączenia po stronie\\ odbiorcy danych\end{tabular} & 1       \\
\hline
\begin{tabular}[c]{@{}l@{}}Uszkodzenie przewodów łączących\\ moduły\end{tabular} & 0.3                & 9         & \begin{tabular}[c]{@{}l@{}}Pomiary nie mieszczące się w normach,\\ brak obrazu\end{tabular}                                               & 3       \\
                                                                                 &                    &           &                                                                                                                                           &        
\end{tabular}
\end{table}
\section{Biblioteki zewnętrzne użyte do zrealizowania projektu}

\begin{itemize}
    \item STM32Cube Low Layer Library - biblioteka STM32
    \item Font ASCII dla Nokia 5110 - SparkFun Electronics
\end{itemize}


%----------------------------------------------------------------------------------------
%	BIBLIOGRAPHY
%----------------------------------------------------------------------------------------
%
%\bibliographystyle{apalike}
%
%\bibliography{sample}
%
%
%----------------------------------------------------------------------------------------


\end{document}